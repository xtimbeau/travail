% Options for packages loaded elsewhere
% Options for packages loaded elsewhere
\PassOptionsToPackage{unicode}{hyperref}
\PassOptionsToPackage{hyphens}{url}
\PassOptionsToPackage{dvipsnames,svgnames,x11names}{xcolor}
%
\documentclass[
  french,
  9pt,
  a4paper,
]{article}
\usepackage{xcolor}
\usepackage[top = 3cm,bottom = 2.5cm,left = 2.5cm,right =
2.5cm]{geometry}
\usepackage{amsmath,amssymb}
\setcounter{secnumdepth}{3}
\usepackage{iftex}
\ifPDFTeX
  \usepackage[T1]{fontenc}
  \usepackage[utf8]{inputenc}
  \usepackage{textcomp} % provide euro and other symbols
\else % if luatex or xetex
  \usepackage{unicode-math} % this also loads fontspec
  \defaultfontfeatures{Scale=MatchLowercase}
  \defaultfontfeatures[\rmfamily]{Ligatures=TeX,Scale=1}
\fi
\usepackage{lmodern}
\ifPDFTeX\else
  % xetex/luatex font selection
\fi
% Use upquote if available, for straight quotes in verbatim environments
\IfFileExists{upquote.sty}{\usepackage{upquote}}{}
\IfFileExists{microtype.sty}{% use microtype if available
  \usepackage[]{microtype}
  \UseMicrotypeSet[protrusion]{basicmath} % disable protrusion for tt fonts
}{}
\usepackage{setspace}
\makeatletter
\@ifundefined{KOMAClassName}{% if non-KOMA class
  \IfFileExists{parskip.sty}{%
    \usepackage{parskip}
  }{% else
    \setlength{\parindent}{0pt}
    \setlength{\parskip}{6pt plus 2pt minus 1pt}}
}{% if KOMA class
  \KOMAoptions{parskip=half}}
\makeatother
% Make \paragraph and \subparagraph free-standing
\makeatletter
\ifx\paragraph\undefined\else
  \let\oldparagraph\paragraph
  \renewcommand{\paragraph}{
    \@ifstar
      \xxxParagraphStar
      \xxxParagraphNoStar
  }
  \newcommand{\xxxParagraphStar}[1]{\oldparagraph*{#1}\mbox{}}
  \newcommand{\xxxParagraphNoStar}[1]{\oldparagraph{#1}\mbox{}}
\fi
\ifx\subparagraph\undefined\else
  \let\oldsubparagraph\subparagraph
  \renewcommand{\subparagraph}{
    \@ifstar
      \xxxSubParagraphStar
      \xxxSubParagraphNoStar
  }
  \newcommand{\xxxSubParagraphStar}[1]{\oldsubparagraph*{#1}\mbox{}}
  \newcommand{\xxxSubParagraphNoStar}[1]{\oldsubparagraph{#1}\mbox{}}
\fi
\makeatother


\usepackage{longtable,booktabs,array}
\usepackage{calc} % for calculating minipage widths
% Correct order of tables after \paragraph or \subparagraph
\usepackage{etoolbox}
\makeatletter
\patchcmd\longtable{\par}{\if@noskipsec\mbox{}\fi\par}{}{}
\makeatother
% Allow footnotes in longtable head/foot
\IfFileExists{footnotehyper.sty}{\usepackage{footnotehyper}}{\usepackage{footnote}}
\makesavenoteenv{longtable}
\usepackage{graphicx}
\makeatletter
\newsavebox\pandoc@box
\newcommand*\pandocbounded[1]{% scales image to fit in text height/width
  \sbox\pandoc@box{#1}%
  \Gscale@div\@tempa{\textheight}{\dimexpr\ht\pandoc@box+\dp\pandoc@box\relax}%
  \Gscale@div\@tempb{\linewidth}{\wd\pandoc@box}%
  \ifdim\@tempb\p@<\@tempa\p@\let\@tempa\@tempb\fi% select the smaller of both
  \ifdim\@tempa\p@<\p@\scalebox{\@tempa}{\usebox\pandoc@box}%
  \else\usebox{\pandoc@box}%
  \fi%
}
% Set default figure placement to htbp
\def\fps@figure{htbp}
\makeatother


% definitions for citeproc citations
\NewDocumentCommand\citeproctext{}{}
\NewDocumentCommand\citeproc{mm}{%
  \begingroup\def\citeproctext{#2}\cite{#1}\endgroup}
\makeatletter
 % allow citations to break across lines
 \let\@cite@ofmt\@firstofone
 % avoid brackets around text for \cite:
 \def\@biblabel#1{}
 \def\@cite#1#2{{#1\if@tempswa , #2\fi}}
\makeatother
\newlength{\cslhangindent}
\setlength{\cslhangindent}{1.5em}
\newlength{\csllabelwidth}
\setlength{\csllabelwidth}{3em}
\newenvironment{CSLReferences}[2] % #1 hanging-indent, #2 entry-spacing
 {\begin{list}{}{%
  \setlength{\itemindent}{0pt}
  \setlength{\leftmargin}{0pt}
  \setlength{\parsep}{0pt}
  % turn on hanging indent if param 1 is 1
  \ifodd #1
   \setlength{\leftmargin}{\cslhangindent}
   \setlength{\itemindent}{-1\cslhangindent}
  \fi
  % set entry spacing
  \setlength{\itemsep}{#2\baselineskip}}}
 {\end{list}}
\usepackage{calc}
\newcommand{\CSLBlock}[1]{\hfill\break\parbox[t]{\linewidth}{\strut\ignorespaces#1\strut}}
\newcommand{\CSLLeftMargin}[1]{\parbox[t]{\csllabelwidth}{\strut#1\strut}}
\newcommand{\CSLRightInline}[1]{\parbox[t]{\linewidth - \csllabelwidth}{\strut#1\strut}}
\newcommand{\CSLIndent}[1]{\hspace{\cslhangindent}#1}

\ifLuaTeX
\usepackage[bidi=basic]{babel}
\else
\usepackage[bidi=default]{babel}
\fi
% get rid of language-specific shorthands (see #6817):
\let\LanguageShortHands\languageshorthands
\def\languageshorthands#1{}


\setlength{\emergencystretch}{3em} % prevent overfull lines

\providecommand{\tightlist}{%
  \setlength{\itemsep}{0pt}\setlength{\parskip}{0pt}}



 


\makeatletter
\@ifpackageloaded{float}{}{\usepackage{float}}
\floatstyle{plain}
\@ifundefined{c@chapter}{\newfloat{apptbl}{h}{lost}}{\newfloat{apptbl}{h}{lost}[chapter]}
\floatname{apptbl}{Tableau A}
\newcommand*\quartoapptblref[1]{Tableau \hyperref[#1]{A\ref{#1}}}
\@ifpackageloaded{caption}{}{\usepackage{caption}}
\DeclareCaptionLabelFormat{quartoapptblreflabelformat}{#1#2}
\captionsetup[apptbl]{labelformat=quartoapptblreflabelformat}
\newcommand*\listofapptbls{\listof{apptbl}{List of Tableaux annexess}}
\makeatother
\makeatletter
\@ifpackageloaded{float}{}{\usepackage{float}}
\floatstyle{plain}
\@ifundefined{c@chapter}{\newfloat{appfig}{h}{lost}}{\newfloat{appfig}{h}{lost}[chapter]}
\floatname{appfig}{Graphique A}
\newcommand*\quartoappfigref[1]{Graphique \hyperref[#1]{A\ref{#1}}}
\@ifpackageloaded{caption}{}{\usepackage{caption}}
\DeclareCaptionLabelFormat{quartoappfigreflabelformat}{#1#2}
\captionsetup[appfig]{labelformat=quartoappfigreflabelformat}
\newcommand*\listofappfigs{\listof{appfig}{List of Graphiques annexess}}
\makeatother
\makeatletter
\@ifpackageloaded{caption}{}{\usepackage{caption}}
\AtBeginDocument{%
\ifdefined\contentsname
  \renewcommand*\contentsname{Table des matières}
\else
  \newcommand\contentsname{Table des matières}
\fi
\ifdefined\listfigurename
  \renewcommand*\listfigurename{Graphiques}
\else
  \newcommand\listfigurename{Graphiques}
\fi
\ifdefined\listtablename
  \renewcommand*\listtablename{Tableaux}
\else
  \newcommand\listtablename{Tableaux}
\fi
\ifdefined\figurename
  \renewcommand*\figurename{Graphique}
\else
  \newcommand\figurename{Graphique}
\fi
\ifdefined\tablename
  \renewcommand*\tablename{Tableau}
\else
  \newcommand\tablename{Tableau}
\fi
}
\@ifpackageloaded{float}{}{\usepackage{float}}
\floatstyle{ruled}
\@ifundefined{c@chapter}{\newfloat{codelisting}{h}{lop}}{\newfloat{codelisting}{h}{lop}[chapter]}
\floatname{codelisting}{Listing}
\newcommand*\listoflistings{\listof{codelisting}{Liste des Listings}}
\makeatother
\makeatletter
\makeatother
\makeatletter
\@ifpackageloaded{caption}{}{\usepackage{caption}}
\@ifpackageloaded{subcaption}{}{\usepackage{subcaption}}
\makeatother
\makeatletter
\@ifpackageloaded{sidenotes}{}{\usepackage{sidenotes}}
\@ifpackageloaded{marginnote}{}{\usepackage{marginnote}}
\makeatother
\makeatletter
\@ifpackageloaded{fontawesome5}{}{\usepackage{fontawesome5}}
\makeatother
\usepackage{bookmark}
\IfFileExists{xurl.sty}{\usepackage{xurl}}{} % add URL line breaks if available
\urlstyle{same}
\hypersetup{
  pdftitle={Une part des salaires dans la VA élévée en 2025 en France},
  pdfauthor={Xavier Timbeau},
  pdflang={fr},
  colorlinks=true,
  linkcolor={blue},
  filecolor={Maroon},
  citecolor={Blue},
  urlcolor={Blue},
  pdfcreator={LaTeX via pandoc}}


%%% title.tex we use to call other packages. Because this works
\usepackage{titlepic}
\usepackage{titling}
\usepackage{graphicx}
\usepackage{fontspec}
\usepackage{placeins}
\usepackage{graphbox}
\usepackage{tikz}
\usepackage{geometry}
\usepackage{xcolor}
\usepackage{amsmath}
\usepackage[some]{background}
\usepackage{lipsum}
\usepackage{caption}
\usepackage{datetime}
\usepackage{xstring}
\usepackage{blindtext}
\usepackage{scrlayer-scrpage}

\setmainfont{OpenSans}[
  UprightFont = {*-Regular},
  BoldFont = {*-Bold},
  BoldItalicFont = {*-BoldItalic},
  ItalicFont = {*-Italic},
  Path = {_extensions/ofce/wp/OpenSans/},
  Extension = {.ttf}
]
\setsansfont{OpenSans}[
  UprightFont = {*-Regular},
  BoldFont = {*-Bold},
  BoldItalicFont = {*-BoldItalic},
  ItalicFont = {*-Italic},
  Path = {_extensions/ofce/wp/OpenSans/},
  Extension = {.ttf}
]

\def\getYear#1{\StrLeft{#1}{4}}
\def\getMonth#1{\StrMid{#1}{6}{7}}
\def\getDay#1{\StrRight{#1}{2}}

\def\jolimois#1{\monthname{\getMonth{#1}}}

\definecolor{quarto-callout-color}{HTML}{eeeeee}
\definecolor{quarto-callout-tip-color}{HTML}{dddddd}
\definecolor{quarto-callout-tip-color-frame}{HTML}{eeeeee}

\clearpairofpagestyles

\KOMAoptions{headsepline=true}

\pagestyle{scrheadings}
\setkomafont{pageheadfoot}{\small}

\rehead{Document de travail n°2025-23}
\lohead{\includegraphics[height=0.25cm]{\_extensions/ofce/wp/ofce\_m.png}}

\lofoot*{\thepage}
\refoot*{\thepage}
\begin{document}



\definecolor{ofcepbbleu}{RGB}{1, 97, 131}
\definecolor{ofcerouge}{RGB}{198, 45, 43}
\definecolor{scporouge}{RGB}{231, 0, 26}

\begin{titlepage}
  \backgroundsetup{
    scale=1,
    angle=0,
    opacity=1,
    contents={
  \begin{tikzpicture}[remember picture,overlay]
    \useasboundingbox (0,0) rectangle(\the\paperwidth,\the\paperheight);
      \node [anchor = center] at (-2.875-5.25,12.5) {\includegraphics[width=2.5cm]{\_extensions/ofce/wp/ofce\_m.png}};
      \node [anchor = center] at (-2.875-5.25,-12.75){\includegraphics[width=2.5cm]{\_extensions/ofce/wp/sciencespo.png}};
      \node [anchor = east] at (8.5,-11.5){\textcolor{black}{Publié le : 2025-09-08}};
      \node [anchor = east] at (8.5,-12){\textcolor{black}{Modifié le : 2025-11-11}};
      \node [anchor = east] at (8.5,13-0.7){\textcolor{gray}{\Huge\textit{Working Paper}}};
      \draw [thick,black](-5.75,-13) -- (-5.75,13);
      \draw [color = white, fill=ofcepbbleu] (8.7,11.4) rectangle (10.45,13.15);
      \draw [color = white, fill=ofcepbbleu, very thick] (8.75-.5,11.25-.5) rectangle (8.75+.6,11.25+.6);
      \node [anchor = east] at (11-0.7,13-0.7){\textcolor{white}{\huge\textbf{23}}};
      \node [anchor = east] at (10-0.74,12-0.67){\textcolor{white}{\textbf{2025}}};
    \end{tikzpicture}
  }
}
\BgThispage

\hspace{4cm}
\begin{minipage}{12.5cm}
  \vspace{5cm}
  \begin{flushleft}
    \begin{spacing}{2.5}
      \textcolor{scporouge}{\Huge\textbf{\textsf{Une part des salaires
dans la VA élévée en 2025 en France}}}
    \end{spacing}
   \vspace{5mm}
  \textcolor{scporouge}{\large\textbf{\textsf{Comparaisons
internationales de 30~ans de partage de la valeur ajoutée}}}
 \vspace{5mm}
    \vspace{20mm}
  \end{flushleft}

  
   \textbf{Xavier Timbeau},           OFCE, Sciences Po Paris
   
  % by-author
 \end{minipage}


%%
%% deuxieme page
\newpage
\pagestyle{empty}
\raisebox{-22cm}{
    \begin{minipage}[b]{40em}
      \textcolor{scporouge}{
        \textbf{CONTACT}}{

          \vspace{0.2cm}
          \vspace{0.2cm}

        \textbf{OFCE}

        10 place de Catalogne

        75014 Paris, FRANCE

        Tel : +33 1 44 18 54 24

        \url{https://www.ofce.sciences-po.fr}
      }
    \end{minipage}
}
%%
%%
\newpage
%% troisieme page
\pagestyle{empty}
%%  \raisebox{3cm}{\begin{minipage}{\linewidth}
%%  \includegraphics[width=2cm]{\_extensions/ofce/wp/ofce\_m.png}
%%
%%  \includegraphics[width=2cm]{\_extensions/ofce/wp/sciencespo.png}
%%  \end{minipage}}
%%

\LARGE\textbf{Une part des salaires dans la VA élévée en 2025 en France}

\large\textbf{Comparaisons internationales de 30~ans de partage de la
valeur ajoutée}

\vspace{1cm}


\par\rule{\textwidth}{0.5pt}

On explore différentes façons de calculer la notion de part des salaires
dans la valeur ajoutée. Le concept privilégié est celui de la part des
salaires \emph{corrigés de la non salarisation} dans la valeur ajoutée
\emph{nette de la consommation de capital fixe} des branches
\emph{marchandes hors services immobiliers}. Il peut être calculé pour
les pays européens. Il fait apparaître une position singulière de la
France où la part des salaires est plus élevé et s'est accrue de façon
importante. Bien que plus fragile empiriquement, le calcul de rendement
net d'impôts du capital productif confirme ce diagnostic particulerement
préoccupant pour le tissu productif français. L'ensemble des éléments
présentés est reproductible à partir des codes fournis.

\par\rule{\textwidth}{0.5pt}


\vspace{1cm}


version en ligne à https://xtimbeau.github.io/travail/


\vspace{1cm}

\begin{flushright}
   \linespread{1}\small{\textbf{Xavier
Timbeau}}, {\small{xavier.timbeau@sciencespo.fr}}\par
 % by-author
\end{flushright}
\vspace{2.5cm}
\begin{minipage}[b]{40em}
  \textcolor{scporouge}{
    \textbf{Remerciements}}{

      \vspace{0.2cm}
      \vspace{0.2cm}

    \textit{\small{Je remercie chaleureusement Elliot Aurissergues,
François Geerolf et Eric Heyer pour leurs remarques pertinentes et
stimulantes. Elles m'ont amené à aller plus loin dans la complexité des
concepts et dans l'exploration des données. La confusion qui en découle
est de ma seule responsabilité.}}
  }
\end{minipage}
\end{titlepage}
\renewcommand*\contentsname{Table des matières}
{
\hypersetup{linkcolor=}
\setcounter{tocdepth}{1}
\tableofcontents
}

\setstretch{1.25}
\marginnote{\begin{footnotesize}

{5379 mots}

\end{footnotesize}}

\section{Du partage de la VA au partage des
richesses}\label{du-partage-de-la-va-au-partage-des-richesses}

L'analyse du partage de la valeur ajoutée (graphique~\ref{fig-psal}) est
au cœur des débats sur la redistribution des richesses (voir notamment
Hurlin et Portier (1996), Timbeau (2002), Cotis (2009), Husson (2010),
Askenazy, Cette et Sylvain (2012), Piton (2019), Timbeau (2025)). Un
indicateur souvent retenu est celui de la part des salaires dans la
valeur ajoutée. Nous discutons ici de la construction de cet indicateur
et de sa comparabilité entre pays (européens ou non).

\begin{figure}[H]

\caption{\label{fig-psal}Part des salaires dans la VA nette}

\centering{

\subcaption{\label{fig-psal-1}Part des salaires dans la VA nette hors
services immobiliers (-L)}

\centering{

\includegraphics[width=1\linewidth,height=\textheight,keepaspectratio]{index_files/figure-pdf/fig-psal-1-1.png}

}

}

\end{figure}%

\subsection{Du bon concept de part des
salaires}\label{du-bon-concept-de-part-des-salaires}

Trois points sont importants pour disposer du bon concept (voir Reis
(2022) pour une discussion et une revue de littérature sur ce point)\,:

\begin{itemize}
\item
  Corriger des non salariés et leur imputer une masse salariale. Cette
  correction est standard. Elle repose sur des hypothèses importantes
  comme le salaire affecté à un non salarié. Nous utilisons la
  décomposition en branches (ou en secteurs dans certains pays) pour
  affecter aux non salariés d'une branche le salaire moyen des salariés
  de cette branche. C'est une hypothèse assez forte, pour laquelle nous
  proposons une alternative, en utilisant la notion de revenu mixte.
  Malheureusement, le revenu mixte par branche\,-- qui est important
  pour le point suivant\,-- n'est pas diffusé systématiquement par les
  Etats membres de l'UE (en fait seule la France diffuse cette donnée).
  Cette correction a des conséquences importante pour les comparaisons
  entre périodes pour un même pays parce que non seulement la part des
  non salariés varie dans le temps, de façon différente suivant les
  branches et entre pays (graphique~\ref{fig-psalcnc}) mais aussi la
  rémunaration des non salariés varie dans le temps (voir l'annexe D
  consacrée à ce point).
\item
  Définir le champ considéré. Il est plus facile de faire le calcul au
  niveau le plus agrégé, mais ce champ inclut les branches non
  marchandes dans lesquelles la notion de prix et donc de valeur ajoutée
  est parfois conventionnelle. Parmi les branches marchandes, la
  branches des services immobiliers est problématique parce qu'elle
  prend en compte la valeur ajoutée des ménages au travers des services
  immobiliers qui sont pour part auto produit (les loyers imputés aux
  propriétaires). La notion de partage de la valeur ajoutée n'a ici pas
  beaucoup de sens et la comparaison d'un pays à l'autre peut être très
  perturbée. La notion privilégiée est donc celle de partage de la
  valeur ajoutée dans les branches marchandes hors services immobiliers
  ou, de façon plus précise, en enlevant de la valeur ajoutée marchande
  la branche «\,services immobiliers (L)\,»
  (\textbf{?@appfig-psalcompote}, graphique~\ref{fig-structbranche}).
  Omettre les services immobiliers pose cependant un problème, parce que
  les loyers versés par les entreprises à leurs bailleurs sont souvent
  un prix de transfert et non un prix de marché. C'est donc un élément
  qui peut être utilisé comme véhicule pour de l'otpimisation fiscale,
  d'autant plus que la fiscalité immobilière peut être avantageuse. Nous
  explorons cette dimension dans la section~\ref{sec-immobilier} en
  particulier pour la France.
\item
  Utiliser la notion de valeur ajoutée nette (de la consommation de
  capital fixe, la CCF) plutôt que brute. Rappelons que la valeur
  ajoutée nette est construite en ôtant de la valeur ajoutée brute la
  consommation de capital fixe. Cette dernière notion découle de
  l'application de tables de mortalité à un inventaire permanent des
  investissements non financiers (i.e.~les investissements physiques
  mais aussi ceux en logiciels ou en base de données ainsi que les
  investissements intangibles comme les marques). En traitant les
  investissements comme une consommation intermédiaire mesurée par leur
  amortissement physique ou fiscal, on est plus proche de la réalité du
  processus productif. Lorsque le taux de dépréciation du capital varie,
  par des changement dans les tables de mortalité, des changements dans
  la composition du capital ou des changements dans la structure par
  branche de l'économie, la CCF rapportée à la valeur ajoutée varie et
  modifie donc la perception des évolutions du partage de la valeur
  ajoutée. La notion de valeur ajoutée nette est meilleure pour des
  comparaisons dans l'espace ou dans le temps. Comme pour la correction
  pour les non salariés, la prise en compte de la valeur ajoutée nette
  modifie dans le temps et dans l'espace la part des salaires dans la
  valeur ajoutée (graphique~\ref{fig-psalnetbrut}).
\end{itemize}

Le concept que nous privilégions est donc défini comme suit (où, pour
chaque branche \(D1_b\) est la masse salariale chargée, \(B1G_b\) la
valeur ajoutée brute, \(P51C_b\) la \(CCF_b\), les trois notions en
euros aux prix courants et \(ns_b\) et \(sal_b\) les effectifs en
personne par branche)\,:

\[
s_{net, n.s., -LOQ}  = \frac{\sum_{b\in{TT-LOQ}}{D1_b*(1+ns_b/sal_b)}}{\sum_{b\in{TT-LOQ}}{B1G_b - P51C_b}}
\]

La part des salaires dans la valeur ajoutée nette est croissante en
France (graphique~\ref{fig-psal}) (de 10~points de 1998 à 2025), comme
en Espagne (de 9~points). Elle atteint en France le niveau le plus élevé
des pays sélectionnés, pour autant que l'on puisse comparer entre pays.

Théoriquement, l'évolution de part des salaires dans la valeur ajoutée
dépend de la fonction de production agrégée (ce qui suppose qu'elle
existe). Si l'élasticité de substitution entre le capital et la travail
est unitaire alors on s'attend à ce que le partage soit indépendant du
prix relatif du travail et du capital. La part des salaires est alors
uniquement déterminée par la forme de la fonction de production et
devrait converger dans tous les pays vers une valeur semblable, par
diffusion de la technologie. Une structure de l'économie par branche
différente peut cependant se traduire par des parts différentes d'un
pays à l'autre.

L'élasticité estimée généralement, au moins à moyen terme, est
sensiblement inférieure à 1, en tout cas sur données macroéconomique.
Cela implique qu'une hausse du prix du travail relativement par au
capital se traduit par une hausse de la part du travail dans la valeur
ajoutée\,-- la réciproque étant bien entendu vraie si c'est le capital
qui est relativement plus cher. Cela peut conduire à des variations plus
persistantes de la part des salaires dans la valeur ajoutée, mais ces
variations doivent reproduire celles des prix relatifs.

La part des salaires dans la valeur ajoutée est la plus basse aux
Pays-Bas et est sur une pente décroissante depuis plus de 20~ans, alors
qu'elle semble stable en Belgique et en Allemagne. L'Italie affiche une
variabilité temporelle importante, avec un pic de la part des salaires
dans la valeur ajoutée en 2013, puis une franche décroissance (de plus
de 13~points) interrompue dans la période récente suite à la période
d'inflation et la forte relance budgétaire.

En France, la hausse est franche après la crise financière de 2008,
suivant une période de grande stabilité de 1995 à 2007. Cette hausse
peut découler d'un effet de structure sectorielle, mais le
graphique~\ref{fig-salaires} indique une autre singularité française.
Contrairement à de nombreux pays, les salaires réels sont restés sur une
pente croissante, interrompue par la phase d'inflation à partir de la
fin de l'année 2021, alors que dans les 5 autres pays, 2008 marque une
cassure dans la progression de salaires réels.

Depuis 2018, en France, la part des salaires est stabilisée, à un haut
niveau (graphique~\ref{fig-psal}). L'inflation et le retard d'ajustement
des salaires sur l'inflation explique probablement cette trajectoire. On
observe des mouvements comparables dans d'autres pays, bien que plus
violent en Allemagne ou en Italie par exemple.

Au début des années 2000, deux pays se distinguaient des autres
(l'Espagne et l'Italie) par une part des salaires plus faibles. L'écart
avec l'Allemagne atteignait alors plus de 15~points. En généralisant
l'approche aux pays de l'Union Européenne, on peut en partie confirmer
cette hypothèse (graphique~\ref{fig-psaleu}). Les pays qui ont connu un
développement rapide, et donc des niveaux d'investissement élevés, on eu
des parts des salaires basses (La Bulgarie, la Tchéquie, la Grèce par
exemple). Mais ce n'est pas une observation systématique\,: certains
pays moins développés ont eu par le passé une part très élevée des
salaires dans la valeur ajouté, témoignant peut être de modes de
formation des salaires et d'inflation particulier et hérités du passé.
Cependant, comme le suggèrent la position singulière de quelques petits
pays, parmi lesquels l'Irlande, le Luxembourg, Malte, Chypre ou les
Pays-Bas dans une certaine mesure, c'est peut être du côté du
déplacement de la base imposable des profits (optimisation fiscale), des
prix de transferts et d'une position très particulière dans la chaîne de
valeur qu'il faut aller chercher l'explication de très faibles parts des
salaires dans la valeur ajoutée.

\begin{figure}[H]

\caption{\label{fig-psaleu}Part des salaires dans la VA nette 1995 et
2025}

\centering{

\subcaption{\label{fig-psaleu-1}Part des salaires dans la VA nette 1995
et 2025 VA nette}

\centering{

\includegraphics[width=1\linewidth,height=\textheight,keepaspectratio]{index_files/figure-pdf/fig-psaleu-1-1.png}

}

}

\end{figure}%

\subsection{Rendement finanicer du capital productif\,: la France au
plus
bas}\label{rendement-finanicer-du-capital-productif-la-france-au-plus-bas}

La construction d'un taux de rendement financier du capital est sans
doute assez fragile parce qu'il faut ajouter à l'évaluation du partage
de la valeur ajoutée une estimation des impôts payés (notamment l'impôt
sur les sociétés) et une évaluation du stock de capital (voir
section~\ref{sec-profits} pour la méthode). En utilisant les données de
stock de capital productif le diagnostic présenté sur le
graphique~\ref{fig-psal} est confirmé par le graphique~\ref{fig-rp}.

Le choix du champ est assez important pour conserver une cohérence entre
numérateur et le dénominateur. Pour le champ hors immobilier, le profit
net et le stock de cpaital excluent toutes les activités immobilières
parce que dans la plupart des pays (Allemagne, France, Italie) seule la
branche immobilière détient des actifs de type logement (\(N111N\) dans
la nomenclature de la NACE rev2). Dans quelques pays, une part
minoritaire de la valeur du stock de logements est détenue par d'autres
branches (F, K, O, Q, R, S en Espagne (15\%) ou en Belgique (0,5\%); K
au Pays-Bas (5\%)) que la branche services immobiliers\,-- ce qui
suggère que la convention branche/secteur n'est pas complètement suivie
ou a été interprétée assez librement.

La France y occupe une position singulière avec un rendement du capital
productif particulièrement faible et décroissant depuis le début des
années 2000 alors qu'il est constant dans beaucoup de pays ou même
croissant comme aux Pays-Bas. Les politiques de l'offre successives
depuis le choc fiscal de Nicolas Sarkozy, le pacte pour la croissance,
la compétitivité et l'emploi de François Hollande en 2012 ou encore les
politiques d'attractivité, en particulier fiscales, engagées par
Emmanuel Macron depuis 2017 n'ont apparemment pas changé grand chose à
cette dégradation continue. Tout au plus, on peut y associer la relative
stabilisation du taux de rendement net en France
(graphique~\ref{fig-rp}) à partir de 2017.

Une explication possible de la dégradation du rendement du capital
productif est à chercher du côté de son accroissement aux Pays-Bas\,--
malgré un poids de l'IS de plus en plus lourd dans ce pays\,-- sous
l'effet du déplacement de la base fiscale à l'intérieur de l'Europe
comme l'analysent Tørsløv, Wier et Zucman (2022) ou encore d'un effet
particulier de la fiscalité immobilière (section~\ref{sec-immobilier}).

\begin{figure}[H]

\caption{\label{fig-rp}Rendement du capital productif (avant et après
IS)}

\centering{

\includegraphics[width=1\linewidth,height=\textheight,keepaspectratio]{index_files/figure-pdf/fig-rp-1.png}

}

\end{figure}%

On détaille en les discutant dans la suite de ce document les effets des
corrections appliquées, ainsi que la différence entre les mesures
dérivées des comptes de branche ou des comptes d'agents (ou de secteurs
institutionnels). Ces éléments sont un peu fastidieux, mais ils
s'avèrent assez importants et pas toujours très intuitifs.

On explore également les conséquences en matière de taux de profit (part
des profits nets dans la valeur ajoutée) ou rendement du capital
(profits nets divisés par les actifs).

\subsection{Salaires réels et
inflation}\label{salaires-ruxe9els-et-inflation}

L'évolution des salaries réels est un complément à celle du partage de
la valeur ajoutée. Pour passer de l'un à l'autre, il faut non seulement
prendre en compte les évolutions de la valeur ajoutée, mais aussi les
effets de l'évolution du ratio prix à la consommation sur prix de valeur
ajoutée.

On déflate la masse salariale (comptabilité nationale, comptes
trimestriels) par les prix à la consommation. On utilise les masses
salariales (\(D1\), dans
\href{https://ec.europa.eu/eurostat/databrowser/view/NAMQ_10_A10/default/table?lang=en}{\texttt{namq\_10\_a10}})
par branches pour comparer branches (principalement) marchandes et
(principalement) non marchandes divisées par l'emploi salarié
(\href{https://ec.europa.eu/eurostat/databrowser/view/NAMQ_10_A10_E__custom_7475124/default/table?lang=en}{\texttt{namq\_10\_a10\_e}}).
Les prix sont les déflateurs de la consommation (\(P31\_S14\) dans
\href{https://ec.europa.eu/eurostat/databrowser/product/page/NAMQ_10_FCS}{\texttt{namq\_10\_fcs}})
chaînés (voir le code pour les détails).

On distingue 4 agrégations\,: l'ensemble des branches (ou l'ensemble de
l'économie), les branches non marchandes, les branches marchandes et les
branches marchandes hors immobilier.

\begin{figure}[H]

\caption{\label{fig-salaires}Salaires réels en Europe}

\centering{

\subcaption{\label{fig-salaires-1}Salaires réels en Europe avec cot.soc.
employeur}

\centering{

\includegraphics[width=1\linewidth,height=\textheight,keepaspectratio]{index_files/figure-pdf/fig-salaires-1-1.png}

}

}

\end{figure}%

En Italie et en Espagne, la masse salariale dans les branches non
marchandes est supérieures à celle des branches non marchandes. Aux
Pays-Bas et en Allemagne il n'y a pas de différence notable. En France,
elle est significativement plus basse. Notons que les branches non
marchandes ne sont pas nécessairement de l'emploi public et ce dans des
proportions variables suivant les pays. Dans tous les pays, la masse
salariale des branches immobilier et (surtout) services financiers est
plus élevée que la masse salariale dans les autres branches marchandes.

\section{Comptes de branches\,: que changent les concepts et le
champ\,?}\label{comptes-de-branches-que-changent-les-concepts-et-le-champ}

\subsection{Salariés et non
salariés}\label{salariuxe9s-et-non-salariuxe9s}

On utilise les données de comptabilité nationale, en trimestriel, par
branche
(\href{https://ec.europa.eu/eurostat/databrowser/view/nasq_10_nf_tr/default/table?lang=en}{\texttt{nasq\_10\_nf\_tr}}),
ré-agrégées au niveau de l'ensemble de l'économie. Le passage par les
comptes de branches permet de distinguer branches marchandes et non
marchandes ou d'autres regroupements, comme l'exclusion des services
immobiliers. Ce passage permet également de conduire la correction
salariés non salariés au niveau des branches. D'après l'INSEE, (voir le
blog
«\,\href{https://blog.insee.fr/combien-pese-l-industrie-en-france-et-en-allemagne/}{Combien
pèse l'industrie en France et en Allemagne}\,»), tous les pays ne
produisent pas une comptabilité de branche mais pour certains (notamment
l'Allemagne) une comptabilité sectorielle. La différence tient aux
entreprises qui produisent plusieurs produits (un constructeur
automobile propose des services financiers pour l'achat des véhicules)
et dont l'activité est imputé à différentes branches (industrie et
services financiers) dans la comptabilité de branche alors que dans une
comptabilité de secteur l'activité est versée dans le principal secteur
(ou le secteur d'immatriculation de l'entreprise chapeau). Cette
différence empêche normalement les comparaisons des comptes de branches
entre pays, y compris à l'intérieur de l'Union Européenne. Cependant,
pour comparer la part des salaries dans la valeur ajoutée sur des
agrégats larges (branches marchandes par exemple), cette dérogation à la
norme comptable n'est que modérément problématique\,: de toute façon,
l'automobile et les services financiers sont agrégés et c'est la
correction pour la masse salariale des non salariés qui peut être
modifiée. Mais si la même délimitation est employée pour les salariés et
les non salariés que pour l'activité, l'erreur est probablement minime.

La part des salaires est corrigée de la part des non salariés (données
annuelles
\href{https://ec.europa.eu/eurostat/databrowser/view/nama_10_a64_e/default/table?lang=en}{\texttt{nama\_10\_a64\_e}},
extrapolées en maintenant le ratio salariés/non salariés à sa dernière
valeur observée) en considérant que le salaire des non salariés est
identique dans chaque branche à celui des salariés\,-- cette hypothèse,
que nous appelons \emph{correction par les effectifs}, sous estime
probablement le salaire de certains des non salariés (notamment les
professions libérales ou les artisans) mais elle est difficile à lever.
Le développement des platerformes et dans certains pays de statuts
(sociaux, fiscaux) particuliers (les microentrepreneurs\,--
autoentrepreneurs en France) a introduit une nouvelle «\,classe\,» de
non salariés possiblement moins rémunérés et avec des durées du travail
plus basses que le reste des indépendants. L'annexe D compile quelques
éléments quantitifs.

En revanche, dans la correction par les effectifs, on prend bien en
compte que les non salariés de la branche agricole n'ont pas le même
revenu que ceux de la branche «\,information et communication\,». La
décomposition employée est à 9 branches et on peut conduire la même
correction à un niveau de désagrégation plus fin. Une alternative est
employée en utilsant les données de revenu mixte. A défaut d'être
totalement convaincante, du fait d'un manque de données diffusées, elle
montre la complexité et l'importance de la correction pour le revenu des
indépendants. Une solution aurait pu être de ne considérer que les
entreprises (voir section~\ref{sec-snff}) mais là aussi les différentes
pratiques de comptabilité ne garantissent pas un traitement homogène
d'un pays à l'autre.

La masse salariale est rapportée soit à la valeur ajoutée brute
(\(B1G\)), soit à la valeur ajoutée nette (\(B1N=B1G-P51C\)). Comme la
consommation de capital fixe (\(P51C\)) n'est pas connue en trimestriel,
elle est dérivée des comptes annuels en 21 branches (niveau 1 de la NACE
rev2 \hyperref[0]{\texttt{nama\_10\_a64}}), agrégée en 9 branches, puis
extrapolée pour les années non connues (ici 2024 et 2025) en conservant
un ratio constant dans la valeur ajouté brute. Le détail se trouve dans
le code.

Les trois graphiques suivants illustrent les conséquences sur la mesure
de la part des salaires suivant les différents concepts. Le
graphique~\ref{fig-psalcnc} compare avec et sans correction pour les non
salariés. Deux rubans sont affichés, l'un pour les branches marchandes
hors services immobiliers et financiers et l'autre pour toutes les
branches.

L'avantage des comptes de branches est une définition homogène pour
chacun des pays. La branche immobilier est exclue parce qu'il est
impossible de distinguer les entreprises des ménages propriétaires (les
loyers imputés sont une valeur ajoutée des ménages).

Les données trimestrielles sont annualisées pour la lisibilité et pour
simplifier le mélange de données annuelles et trimestrielles. Le point
2025 (la dernière année) est donc un acquis sur les trimestres observés
de l'année (ici 2 trimestres sur 4) susceptible de changer au fur et à
mesure du temps. Il est possible en modifiant le code de produire un
graphique trimestriel ou trimestriel lissé, à votre convenance.

\begin{figure}[H]

\caption{\label{fig-psalcnc}Correction pour la non salarisation, comptes
de branches}

\centering{

\includegraphics[width=1\linewidth,height=\textheight,keepaspectratio]{index_files/figure-pdf/fig-psalcnc-1.png}

}

\end{figure}%

La correction de la non salarisation, en imputant une masse salariale
pour les entrepreneurs individuels à partir de la rémunération moyenne
des salariés, augmente la part des salaires. La correction n'est pas
constante dans le temps (c'est particulièrement fort pour la France) ni
dans l'espace (la correction est très forte en Italie). La correction
est plus importante lorsqu'on se limite aux branches marchandes hors
services immobiliers et services financiers, sauf aux Pays-Bas.

Cette convention d'imputation d'un salaire aux non salariés est assez
habituelle. Elle pose cependant un problème de cohérence avec la notion
de revenu mixte mesuré dans la comptabilité nationale. Le revenu mixte
est le revenu des entrepreneurs individuels, autoentrepreneurs et autres
catégories d'indépendants. Il n'intègre pas l'activité des ménages en
tant qu'employeurs (employés de maison) ou en tant que bailleur (loyers
perçus ou imputés) qui sont comptabilisés dans l'excédent brut
d'exploitation des ménages (S14). Le tableau~\ref{tbl-mixte} résume pour
les 6 principaux pays analysés l'écart entre masse salariale imputée (en
2024) et revenu mixte.

\begin{table}

\caption{\label{tbl-mixte}Revenu mixte et masse salariale imputée aux
non salariés}

\centering{

\fontsize{9.0pt}{11.0pt}\selectfont
\begin{tabular*}{\linewidth}{@{\extracolsep{\fill}}l>{\raggedleft\arraybackslash}p{\dimexpr 75.00pt -2\tabcolsep-1.5\arrayrulewidth}>{\raggedleft\arraybackslash}p{\dimexpr 75.00pt -2\tabcolsep-1.5\arrayrulewidth}>{\raggedleft\arraybackslash}p{\dimexpr 75.00pt -2\tabcolsep-1.5\arrayrulewidth}}
\toprule
 & (a) Salaires des non sal. & (b) Revenu mixte & (a)/(b) \\ 
\midrule\addlinespace[2.5pt]
Allemagne & 195 Mds€  & 324 Mds€  &  60\%  \\ 
France & 181 Mds€  & 138 Mds€  & 131\%  \\ 
Italie & 240 Mds€  & 270 Mds€  &  89\%  \\ 
Espagne & 119 Mds€  & 159 Mds€  &  75\%  \\ 
Pays-Bas & 106 Mds€  & 103 Mds€  & 103\%  \\ 
Belgique &  56 Mds€  &  40 Mds€  & 142\%  \\ 
\bottomrule
\end{tabular*}
\begin{minipage}{\linewidth}
\emph{Champ} : Branches marchandes et non marchandes, année 2024 pour (a) et (b).\\
\emph{Source} : Eurostat, comptes annuels de branche (nama\_10\_a64), et de transactions non financières (nasa\_10\_nf\_tr).\\
\end{minipage}

}

\end{table}%

En France et en Belgique, la convention employée surestime probablement
la masse salariale que l'on peut imputer aux non salariés, puisqu'elle
apparaît supérieure au revenu mixte. En revanche, en Allemagne ou en
Espagne, c'est potentiellement l'inverse. Le cas Français est
intéressant, parce que le ratio masse salariale imputée sur revenu mixte
devient plus grand que 1 en 2013 et et augmente quelques années avant.
Le régime d'autoentrepreneur introduit en 2008 peut expliquer cette
dynamique en augmentant le nombre de non salariés ayant de faibles
revenus et éventuellement étant salariés par ailleurs (voir l'annexe D
pour plus de détails).

On peut construire à partir du revenu mixte une correction alternative
de la non salarisation en utilisant le revenu mixte\,-- en retenant 88\%
du revenu mixte\footnote{Le chiffre de 88\% est établi à partir des
  comptes nationaux français qui permettent de calculer le ratio CCF sur
  revenu mixte pour les entrepreneurs individuels (secteur
  institutionnel S14AA). Pour la France nous utilisons le ratio calculé
  chaque année (qui varie entre 9,4\% et 13,8\% entre 1995 et 2024). Ce
  chiffre est plus faible que celui pour l'ensemble de l'économie
  marchande (de l'ordre de 20\%, voir l'annexe B), ce qui peut
  s'expliquer par le fait que les structures fortement capitalistiques
  (et avec une forte CCF) sont moins suceptibles d'être des indépedants,
  parce qu'il est nécessaire de les financer par une structure de
  capital à responsabilité limitée et faisant intervenir des
  actionnaires qui ne sont pas le seul dirigeant. Enfin, nous supposons
  que dans le cas des indépendants la part des profits est nulle et que
  la rémunération de l'indépendant est équivalente à un revenu
  d'activité, ce qui est conforme au traitement fiscal en général.
  Lorsque l'entrepreneur souhaite bénéficier d'une fiscalité différente
  sur les revenus d'activité et les revenus du capital, il est contraint
  de passer d'un régime d'indépendant à une structure juridique moins
  souple.}, le reste pour approcher la CCF\,-- et non le nombre de non
salariés (correction par les effectifs) comme base de la correction.

\begin{figure}[H]

\caption{\label{fig-psalmixte}Correction pour la non salarisation,
effectifs ou revenu mixte, comptes de branches}

\centering{

\includegraphics[width=1\linewidth,height=\textheight,keepaspectratio]{index_files/figure-pdf/fig-psalmixte-1.png}

}

\end{figure}%

La correction est assez importante pour la France et la Belgique, ce qui
est conforme aux valeurs affichées dans le tableau~\ref{tbl-mixte}. La
hausse de la part des salaires apparaît moins importante que pour la
correction par les effectifs. Au lieu d'une hausse de 8,3~points de
pourcentage, elle n'est que de de 4,4~points. La comparaison avec les
autres pays est donc un peu moins brutale que pour la correction par les
effectifs, l'Allemagne ayant en 2023 une part des salaires dans la
valeur ajoutée proche de celle de la France en suivant la correction par
le revenu mixte, qui réhausse la part des salaires (voir aussi le
tableau~\ref{tbl-mixte})\,-- à la réserve près que la correction est
moins précise pour l'Allemagne puisque nous n'avons pas pu accéder aux
données de revenu mixte par branches, contrairement à la France.

\subsection{Valeur ajoutée nette ou
brute}\label{valeur-ajoutuxe9e-nette-ou-brute}

\begin{figure}[H]

\caption{\label{fig-psalnetbrut}VA Nette ou brute, comptes de branches}

\centering{

\includegraphics[width=1\linewidth,height=\textheight,keepaspectratio]{index_files/figure-pdf/fig-psalnetbrut-1.png}

}

\end{figure}%

La notion de part des salaires dans la valeur ajoutée nette consiste à
réduire le démominateur (la valeur ajoutée) de la consommation de
capital fixe. Cela augmente donc le ratio. Cependant, cette correction
n'est pas constante dans le temps (comme en France, en Espagne ou en
Belgique). Comme on peut le voir sur le \textbf{?@appfig-psalcompote},
la variance entre les pays est plus basse pour la notion brute (non
corrigé de la CCF) que nette. Pour les branches marchandes hors services
immobiliers et services immobiliers, le classement entre pays est
marginalement modifié, la Belgique ayant une part des salaires nette pus
élevée que l'Allemagne, alors que sa part brute est plus faible qu'en
Allemagne. Pour les autres pays, le classement est indentique (La France
a la part la plus haute et les Pays-Bas plus faible).

\subsection{Impact du changement de structure de
l'économie}\label{impact-du-changement-de-structure-de-luxe9conomie}

On peut décomposer le changement de la part des salaires dans la valeur
ajoutée en un effet de structure en branche et un effet de changement de
la part des salaires dans la valeur ajoutée dans chaque branche.
Formellement la décomposition retenue s'écrit (où \(w_{b,t}\) est la
part de VAN de la branche \(b\) dans la valeur ajoutée nette de
l'ensemble des branches considérées et \(s_{b,t}\) la part des salaires
dans la branche \(b\))\,:

\[
s_t - \sum w_{b,1995} \times s_{b,1995} =  \sum w_{b,1995} \times (s_{b,t}-s_{b,1995}) + \sum (w_{b,t} - w_{b,1995}) \times s_{b,t}  
\]

L'année 1995 est l'année de référence et le premier terme (de droite)
s'interprète comme la part des salaires qui prévaudrait s'il n'y avait
pas eu de changement de structure. Le graphique~\ref{fig-structbranche}
représente ce terme ainsi que la part agrégée des salaires (\(s_{t}\)).
Leffet de la structure par branche de l'économie (ici marchande hors
services immobiliers produits par les ménages) est assez marginale. Les
variations de la part des salaires sont bien celle des parts des
salaires dans chaque secteur.

Il existe quelques exceptions à cette régle générale. A structure de
branche inchangée, avec comme année de référence 1995, la part des
salaires serait plus basse de 3,5~points de VA pour les Pays-Bas en
2025. En Allemagne ou en Belgique, le changement de structure des
branches explique un petit peu de l'évolution à la hausse.

En revanche, la part des salaires serait légèrement supérieure en Italie
à structure inchangée. Le pic de valeur ajoutée en 2013 est lié
entièrement à la structure par branche, ce qui laisse supposer une
rupture de série dans les comptes de branche.

\begin{figure}[H]

\caption{\label{fig-structbranche}Structure par branche et part des
salaires dans la VA}

\centering{

\includegraphics[width=1\linewidth,height=\textheight,keepaspectratio]{index_files/figure-pdf/fig-structbranche-1.png}

}

\end{figure}%

\subsection{Profits nets dans les comptes de branche}\label{sec-profits}

En utilisant d'une part la valeur ajoutée des branches marchandes hors
services immobiliers produits par les ménages et la valeur des actifs
productifs issue de la base des stocks de capital productif
(\href{https://ec.europa.eu/eurostat/databrowser/view/NAMA_10_NFA_ST__custom_2046386/default/table?lang=en}{nama\_10\_nfa\_st})
sur le même champ (i.e.~en enlevant la sous branche L68A), on peut
estimer un rendement du capital productif.

\begin{figure}[H]

\caption{\label{fig-tprofitbranches}Rendements du capital, compte de
branches}

\centering{

\subcaption{\label{fig-tprofitbranches-1}Rendements du capital, compte
de branches Marchandes-L}

\centering{

\includegraphics[width=1\linewidth,height=\textheight,keepaspectratio]{index_files/figure-pdf/fig-tprofitbranches-1-1.png}

}

}

\end{figure}%

Le rendement calculé sur le graphique~\ref{fig-tprofitbranches} diffère
de celui du graphique~\ref{fig-tprofitsnff}. La différence vient en
partie de la difficulté à imputer l'impôt des sociétés aux seules
entreprises des branches marchandes\,-- des entités légales dans les
branches non marchandes peuvent être soumises à l'impôt sur les sociétés
et de la valorisation des actifs. Dans l'approche comptes d'agents (ou
de secteurs institutionnels), on affecte la valeur nette résiduelle des
entreprises au stock de capital. De plus, le stock de capital n'est pas
dans l'approche du graphique~\ref{fig-tprofitsnff} limité au capital
productif mais intègre également des actifs financiers qui n'ont pas de
contrepartie physique parce qu'ils sont hors territoire français.

La France conserve une singularité marquée par la baisse continue du
taux de profit au cours du temps. Le rendement apparent du capital est
ainsi très bas, plus bas que dans tous les autres pays considérés où il
est plutôt stable (l'Italie fait exception avec une forte volatilité).
Cette singularité subsiste lorsqu'on utilise la correction de la non
salarisation par le revenu mixte au lieu des comptes de branches
(graphique~\ref{fig-rpmixte}). La position de la France est un peu moins
singulière, mais le diagnostic subsiste. L'écart avec l'Allemagne est
significativement réduit dans cette configuration, puisqu'en 2023 le
rendement du capital productif ressort à 8,2\% en France contre 9,3\% en
Allemagne. L'écart avec les Pays-Bas subsiste presque entièrement
(19,2\% de rendement aux Pays-Bas).

\begin{figure}[H]

\caption{\label{fig-rpmixte}Rendement du capital productif, correction
de la non salarisation par les effectifs ou le revenu mixte}

\centering{

\includegraphics[width=1\linewidth,height=\textheight,keepaspectratio]{index_files/figure-pdf/fig-rpmixte-1.png}

}

\end{figure}%

Reis (2022) conclut que le rendement du capital productif est plutôt
constant au cours du temps, pour les 20 dernières années. L'analyse
présentée ici le contredit pour la France et possiblement d'autres pays.
Les données utilisées ne sont pas les mêmes, puisqu'il utilise
principalement AMECO et que l'analyse conduite ici exploite plus de
profondeur dans les données de comptabilité nationale.

\section{Comptes d'agents ou de secteurs institutionnels\,: Entreprise
non financières et financières}\label{sec-snff}

\subsection{Part des salaires dans la valeur ajoutée, comptes
d'agents}\label{part-des-salaires-dans-la-valeur-ajoutuxe9e-comptes-dagents}

Les comptes d'agents (ou de secteurs institutionnels) permettent une
analyse plus simple, parce qu'ils permettent de distinguer les seules
entreprises non financières. Cela évite d'avoir à prendre en compte les
non salariés, cela exclue les services immobiliers produits par les
ménages. C'est donc une analyse sur un champ économique plus strict (au
sens de la forme légale des entités considérées). La notion d'impôt sur
les sociétés est aussi mieux définie et le stock de capital productif
est mieux connu du fait de l'obligation légale de déclaration des
comptes des entreprises.

Malheureusement, comme identifié par l'INSEE, la pratique des instituts
nationaux européens n'est pas conforme à celle de l'INSEE. Par exemple,
en Allemagne, le secteur S11 inclut les quasi-sociétés et les
entrepreneurs individuels. La normalisation des concepts est par
ailleurs peu probable dans le futur, puisqu'elle est liée aux pratiques
administratives.

Comme pour les graphiques précédents, les données trimestrielles sont
annualisées (pour éliminer la variabilité trimestrielle qui nuit à la
lisibilité et qui n'a pas beaucoup de sens). En trait pointillé, on
représente la part de la valeur ajoutée dans les branches marchandes
hors immobilier et corrigée de la non salarisation pour mesurer la
différence des concepts.

\begin{figure}[H]

\caption{\label{fig-s11psal}Part des salaires dans la VA, SNF, comptes
d'agents}

\centering{

\includegraphics[width=1\linewidth,height=\textheight,keepaspectratio]{index_files/figure-pdf/fig-s11psal-1.png}

}

\end{figure}%

\begin{figure}[H]

\caption{\label{fig-s1112psal}Part des salaires dans la VA, SNF+SF,
comptes d'agents}

\centering{

\includegraphics[width=1\linewidth,height=\textheight,keepaspectratio]{index_files/figure-pdf/fig-s1112psal-1.png}

}

\end{figure}%

\subsection{Profits nets et dividendes dans les comptes
d'agents}\label{sec-profitagent}

Les comptes des sociétés non financières permettent d'examiner d'autres
éléments du compte. On affiche ici le profit net sur la valeur ajoutée
nette, et le taux de dividendes nets sur la valeur ajoutée nette.

Les profits nets sont définis comme la valeur ajoutée nette de la
consommation de capital fixe moins la rémunération des salariés, moins
les taxes nettes des subventions moins l'impôt sur les sociétés\,:

\[
\Pi = B1G - P51C - D1 - (D2-D3) - D5 = B2N-D5
\]

Les dividendes sont la ligne \(D42\) nette de ce qui est payé et reçu
par le secteur des sociétés non financières (SNF ou S11).

\begin{figure}[H]

\caption{\label{fig-profits}Profits nets dans la VA, SNF, comptes
d'agents}

\centering{

\includegraphics[width=1\linewidth,height=\textheight,keepaspectratio]{index_files/figure-pdf/fig-profits-1.png}

}

\end{figure}%

On peut rapporter ces notions aux éléments qui viennent du compte de
capital. Le premier concept est le profit rapporté au stock de capital
physique (tel que valorisé dans la comptabilité nationale, c'est-à-dire
à la valeur de remplacement et au prix de marché). Malheureusement, à
part la France, aucun pays dans notre échantillon ne diffuse ces données
sur Eurostat.

On rapporte également à une notion financière, à savoir la valeur nette
des actions au passif des comptes d'entreprises. Les conventions de
valorisation des actions non cotées sont délicates et difficiles à
suivre d'un pays à l'autre. On choisit ici d'augmenter ces actions de la
valeur nette résiduelle des entreprises non financières (\(BF90\)). On a
donc\,:

\[
\begin{aligned}
r_{productif} & = \frac{\Pi}{N1N+N2N} \\
r_{financier} & = \frac{\Pi}{F51+F52+BF90}
\end{aligned}
\]

On obtient le graphe suivant\,:

\begin{figure}[H]

\caption{\label{fig-tprofitsnff}Rendements du capital, compte de
secteur}

\centering{

\subcaption{\label{fig-tprofitsnff-1}Rendements du capital, compte de
secteur SNF}

\centering{

\includegraphics[width=1\linewidth,height=\textheight,keepaspectratio]{index_files/figure-pdf/fig-tprofitsnff-1-1.png}

}

}

\end{figure}%

\section{Au delà de l'Europe et pour l'ensemble de
l'économie}\label{au-deluxe0-de-leurope-et-pour-lensemble-de-luxe9conomie}

L'accès aux données de l'OCDE est devenu particulièrement opaque, mais
je m'en suis sorti. Il est possible d'utiliser des données de
comptabilité nationale, au niveau de l'ensemble de l'économie (y compris
donc les branches non marchandes et l'immobilier). La correction pour la
non-salarisation est assurée par les données de l'\emph{Economic
Outlook} (avec une trimestrialisation ad hoc). Au lieu de la valeur
ajoutée, on utilise le PIB, auquel on enlève la consommation de capital
fixe (dans les données OCDE, il n'y a pas de données de CCF pour le
Japon publiées). Le concept de part des salaires n'est donc pas tout à
fait le même que dans les autres analyses.

On obtient le graphique~\ref{fig-psaloecd}, qui rejoint ceux présentés,
bien que l'agrégation à l'ensemble de l'économie écrase les évolutions.

\begin{figure}[H]

\caption{\label{fig-psaloecd}Part des salaries dans le PIN, données SNA
de l'OCDE}

\centering{

\includegraphics[width=1\linewidth,height=\textheight,keepaspectratio]{index_files/figure-pdf/fig-psaloecd-1.png}

}

\end{figure}%

En revanche, Eurostat a un programme de coopération (avec l'OCDE) pour
intégrer les données dans le cadre d'Eurostat (sans que les méthodes ne
soient plus homogènes pour autant).

On utilise ces données
(\href{https://ec.europa.eu/eurostat/databrowser/view/naidsa_10_nf_tr__custom_13241966/default/table?lang=en}{\texttt{naidsa\_10\_nf\_tr}})
pour construire des indicateurs comparables en comparant des pays autres
que ceux de la zone euro.

\begin{figure}[H]

\caption{\label{fig-tprofithze}Part des salaires dans la valeur ajoutée,
comptes de secteur, comparaisons internationales}

\centering{

\includegraphics[width=1\linewidth,height=\textheight,keepaspectratio]{index_files/figure-pdf/fig-tprofithze-1.png}

}

\end{figure}%

\section{Le rendement de l'immobilier}\label{sec-immobilier}

Plusieurs explications peuvent être avancées à la dégradation du
rendement du capital productif\footnote{On peut aussi mettre en cause la
  qualité des comptes nationaux et leur incapacité à apporter une
  information même bruitée sur la réalité des économies contemporaines.
  Cette explication est commode lorsqu'on a une théorie envers laquelles
  les «\,faits\,» comptables sont têtus. Nous partons du principe qu'on
  peut} :

\begin{itemize}
\item
  une part des salaires dans la valeur ajoutée trop importante, ce que
  suggèrent les éléments présentés plus haut,
\item
  une fonction de production agrégée spécifique qui implique une part de
  l'immobilier plus importante compte tenu de la structure et de la
  technologie de l'économie française. Alternativement, si les services
  immobiliers sont imparfaitement substituables (ce qui est probable)
  aux autres facteurs de production, un prix relatif plus élévé de ces
  services immobiliers peut conduire à une part plus importante des
  consommations intermédiaires (en valeur).
\item
  une mauvaise évaluation de la valeur des actifs productifs conduisant
  à surestimer le stock de capital productif, par exemple en sous
  estimant la consommation de capital fixe. L'annexe B tend à contredire
  cette idée, la CCF étant plutôt plus élevée en France qu'ailleurs et
  le taux de dépréciation (CCF sur actif) étant assez stable dans le
  temps.
\item
  un mécanisme d'optimisation fiscale par des prix de transfert vers
  d'autres pays européens (ce qu'explorent Tørsløv, Wier et Zucman
  (2022)), en particulier les Pays-Bas, qui affichent un rendement élevé
  et croissant du capital productif,
\item
  un autre mécanisme d'optimisation fiscale, par la séparation des
  activités productives et des locaux qui leur sont nécessaires. Les
  loyers sont alors un prix de transfert et permettraient de bénéficier
  de la fiscalité avantageuse de l'immobilier\footnote{Une recherche sur
    internet aboutit rapidement à des documents de ce type\,:
    \href{https://www.frenchfigures.com/blog/placement-immobilier-entreprise\#:~:text=Optimisation\%20de\%20l\%27Imp\%C3\%B4t\%20sur\%20les\%20Soci\%C3\%A9t\%C3\%A9s&text=Les\%20loyers\%20vers\%C3\%A9s\%20par\%20votre\%20entreprise\%20\%C3\%A0\%20la\%20structure\%20d\%C3\%A9tenant,patrimoine\%20immobilier\%20du\%20risque\%20entrepreneurial.}{Les
    fondamentaux de l'investissement immobilier professionnel}. Le
    principe est que les plus values immobilières sont moins taxées que
    les flux de revenus à terme.}.
\end{itemize}

Nous explorons dans cette section cette piste. La difficulté est que
l'activité résidentielle, et sa partie autoproduite, sont intégrées dans
la branche «\,services immobiliers\,». A partir des données détaillées
de consommation (base
\href{https://ec.europa.eu/eurostat/databrowser/view/nama_10_cp18__custom_18526061/default/table}{\texttt{nama\_10\_cp18}}),
on peut approximativement\footnote{En ne prenant en compte que la
  consommation, on manque l'investissement des ménages en services
  immobiliers, à savoir les frais de transaction et d'agence lors des
  ventes.} reconstituer ces parts. Le tableau~\ref{tbl-households} fait
apparaître que la valeur ajoutée de la branche «\,services immobiliers
(L)\,» est particulièrement importante en France et que la partie hors
service de logement des ménages est aussi particulièrement élevée.

\begin{table}

\caption{\label{tbl-households}Part des loyers et des loyers imputés
dans la valeur ajoutée marchande}

\centering{

\fontsize{9.0pt}{11.0pt}\selectfont
\begin{tabular*}{\linewidth}{@{\extracolsep{\fill}}>{\centering\arraybackslash}p{\dimexpr 22.50pt -2\tabcolsep-1.5\arrayrulewidth}>{\raggedright\arraybackslash}p{\dimexpr 90.00pt -2\tabcolsep-1.5\arrayrulewidth}>{\raggedleft\arraybackslash}p{\dimexpr 75.00pt -2\tabcolsep-1.5\arrayrulewidth}>{\raggedleft\arraybackslash}p{\dimexpr 75.00pt -2\tabcolsep-1.5\arrayrulewidth}>{\raggedleft\arraybackslash}p{\dimexpr 75.00pt -2\tabcolsep-1.5\arrayrulewidth}>{\raggedleft\arraybackslash}p{\dimexpr 75.00pt -2\tabcolsep-1.5\arrayrulewidth}}
\toprule
 &  & \multicolumn{4}{>{\centering\arraybackslash}m{\dimexpr 300.00pt -2\tabcolsep-1.5\arrayrulewidth}}{Part dans la VAB marchande en 2024 des} \\ 
\cmidrule(lr){3-6}
 &  & services immobiliers & services immobiliers hors m\\'enages & loyers des m\\'enages & loyers imput\\'es aux m\\'enages \\ 
\midrule\addlinespace[2.5pt]
1 & Allemagne & 12,3\% & 1,8\% & 10,5\% &  5,6\% \\ 
2 & France & 18,6\% & 4,0\% & 14,5\% & 10,8\% \\ 
3 & Italie & 15,5\% & 2,8\% & 12,7\% & 10,8\% \\ 
4 & Espagne & 13,9\% & 2,4\% & 11,5\% &  9,2\% \\ 
5 & Pays-Bas & 10,1\% & 0,1\% &  9,9\% &  6,9\% \\ 
6 & Belgique & 12,9\% & 0,3\% & 12,6\% &  8,3\% \\ 
\bottomrule
\end{tabular*}
\begin{minipage}{\linewidth}
*Source *: Eurostat Household final consumption expenditure by purpose (nama\_10\_cp18)\\
\end{minipage}

}

\end{table}%

Le graphique~\ref{fig-ciLfr}, pour la France uniquement, caractérise
l'évolution de la consommation en services immobiliers par le secteur
productif hors services aux ménages. La hausse est continue depuis le
début de la période d'observation (qui débute en 1978 sur le
graphique~\ref{fig-ciLfr}), et s'achève en 2006. Cette hausse de la part
des consommations intermédiaires ne peut donc pas expliquer à elle seule
les évolutions du rendement apparent du capital productif.

\begin{figure}[H]

\caption{\label{fig-ciLfr}Consommations intermédiaires en services
immobiliers, France}

\centering{

\includegraphics[width=1\linewidth,height=\textheight,keepaspectratio]{index_files/figure-pdf/fig-ciLfr-1.png}

}

\end{figure}%

Le graphique~\ref{fig-rendfr} confirme cette idée, mais illustre l'ordre
de grandeur que les services immobiliers pourraient avoir sur le
rendement du capital. Les deux courbes présentées sont d'une part le
rendement avant IS du capital productif des branches marchandes non
immobilières et d'autre part le même rendement, mais en modifiant la
valeur ajoutée des branches marchandes non immobilières en appliquant le
ratio consommation intermédiaire en services immobiliers de 1980 (il
était alors de 3,3\%). Cela représente plus d'un point de rendement.

\begin{figure}[H]

\caption{\label{fig-rendfr}Rendement du capital productif, sensibilité
aux CI en services immobiliers, France}

\centering{

\includegraphics[width=1\linewidth,height=\textheight,keepaspectratio]{index_files/figure-pdf/fig-rendfr-1.png}

}

\end{figure}%

\faIcon{square}

\section*{Références}\label{ruxe9fuxe9rences}
\addcontentsline{toc}{section}{Références}

\phantomsection\label{refs}
\begin{CSLReferences}{0}{1}
\bibitem[\citeproctext]{ref-askenazy2012}
Askenazy P., Cette G., Sylvain A. (2012).
{«~\href{https://doi.org/10.3917/dec.asken.2012.01}{Le partage de la
valeur ajoutée}~»}, \emph{Repères}.

\bibitem[\citeproctext]{ref-cotis2009}
Cotis J.-P. (2009).
{«~\href{https://www.vie-publique.fr/rapport/30455-partage-valeur-ajoutee-partage-profits-et-ecarts-de-remuneration}{Partage
de la valeur ajoutée, partage des profits et écarts de rémunérations en
France}~»}, Présidence de la République Française.

\bibitem[\citeproctext]{ref-hurlin1996}
Hurlin C., Portier F. (1996).
{«~\href{https://doi.org/10.3406/ecop.1996.5809}{Le partage de la valeur
ajoutée dans le cycle}~»}, \emph{Économie \& prévision}, \emph{125}, n°
4, p.~73‑85.

\bibitem[\citeproctext]{ref-husson2010}
Husson M. (2010). {«~\href{https://doi.org/10.3917/rdli.064.0047}{Le
partage de la valeur ajoutée en Europe}~»}, \emph{La Revue de l'Ires},
\emph{n° 64}, n° 1, p.~47‑91.

\bibitem[\citeproctext]{ref-piton2019}
Piton S. (2019). {«~\href{https://doi.org/10.3917/rce.024.0131}{7. Le
partage de la valeur ajoutée~n{'}a pas encore dévoilé tous ses
mystères}~»}, \emph{Regards croisés sur l'économie}, \emph{n° 24}, n° 1,
p.~131‑140.

\bibitem[\citeproctext]{ref-reis2022}
Reis R. (2022).
{«~\href{https://personal.lse.ac.uk/reisr/papers/99-ampf.pdf}{Which
r-star, public bonds or private investment? Measurement and policy
implications.}~»},.

\bibitem[\citeproctext]{ref-timbeau2002}
Timbeau X. (2002). {«~\href{https://doi.org/10.3917/reof.080.0063}{Le
partage de la valeur ajoutée en France}~»}, \emph{Revue de l'OFCE},
\emph{80}, n° 1, p.~63.

\bibitem[\citeproctext]{ref-timbeau2025}
Timbeau X. (2025).
{«~\href{https://shs.cairn.info/revue-l-economie-politique-2025-1-page-64?lang=fr}{Quelles
marges de manœuvre pour revaloriser le travail ?}~»}, \emph{Economie
Politique}, \emph{105}, n° 1, p.~64‑75.

\bibitem[\citeproctext]{ref-tuxf8rsluxf8v2022}
Tørsløv T., Wier L., Zucman G. (2022).
{«~\href{https://doi.org/10.1093/restud/rdac049}{The Missing Profits of
Nations}~»}, \emph{The Review of Economic Studies}, \emph{90}, n° 3,
p.~1499‑1534.

\end{CSLReferences}

\section*{Suppléments}\label{fappfig-psalcompote}
\addcontentsline{toc}{section}{Suppléments}

6 annexes sont accessibles en ligne à l'adresse
\url{https://xtimbeau.github.io/travail/} :

\begin{itemize}
\tightlist
\item
  annexe A\,: Domaine des données,
\item
  annexe B\,: CCF,
\item
  annexe C\,: comparaisons entre pays,
\item
  annexe D\,: rémunération des non salariés,
\item
  annexe E\,: CN2020 et CN2014 pour la France,
\item
  annexe F\,: Rendement du capital productif de 1978 à 2024)
\end{itemize}




\end{document}
